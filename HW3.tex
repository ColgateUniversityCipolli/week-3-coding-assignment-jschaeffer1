\documentclass{article}\usepackage[]{graphicx}\usepackage[]{xcolor}
% maxwidth is the original width if it is less than linewidth
% otherwise use linewidth (to make sure the graphics do not exceed the margin)
\makeatletter
\def\maxwidth{ %
  \ifdim\Gin@nat@width>\linewidth
    \linewidth
  \else
    \Gin@nat@width
  \fi
}
\makeatother

\definecolor{fgcolor}{rgb}{0.345, 0.345, 0.345}
\newcommand{\hlnum}[1]{\textcolor[rgb]{0.686,0.059,0.569}{#1}}%
\newcommand{\hlsng}[1]{\textcolor[rgb]{0.192,0.494,0.8}{#1}}%
\newcommand{\hlcom}[1]{\textcolor[rgb]{0.678,0.584,0.686}{\textit{#1}}}%
\newcommand{\hlopt}[1]{\textcolor[rgb]{0,0,0}{#1}}%
\newcommand{\hldef}[1]{\textcolor[rgb]{0.345,0.345,0.345}{#1}}%
\newcommand{\hlkwa}[1]{\textcolor[rgb]{0.161,0.373,0.58}{\textbf{#1}}}%
\newcommand{\hlkwb}[1]{\textcolor[rgb]{0.69,0.353,0.396}{#1}}%
\newcommand{\hlkwc}[1]{\textcolor[rgb]{0.333,0.667,0.333}{#1}}%
\newcommand{\hlkwd}[1]{\textcolor[rgb]{0.737,0.353,0.396}{\textbf{#1}}}%
\let\hlipl\hlkwb

\usepackage{framed}
\makeatletter
\newenvironment{kframe}{%
 \def\at@end@of@kframe{}%
 \ifinner\ifhmode%
  \def\at@end@of@kframe{\end{minipage}}%
  \begin{minipage}{\columnwidth}%
 \fi\fi%
 \def\FrameCommand##1{\hskip\@totalleftmargin \hskip-\fboxsep
 \colorbox{shadecolor}{##1}\hskip-\fboxsep
     % There is no \\@totalrightmargin, so:
     \hskip-\linewidth \hskip-\@totalleftmargin \hskip\columnwidth}%
 \MakeFramed {\advance\hsize-\width
   \@totalleftmargin\z@ \linewidth\hsize
   \@setminipage}}%
 {\par\unskip\endMakeFramed%
 \at@end@of@kframe}
\makeatother

\definecolor{shadecolor}{rgb}{.97, .97, .97}
\definecolor{messagecolor}{rgb}{0, 0, 0}
\definecolor{warningcolor}{rgb}{1, 0, 1}
\definecolor{errorcolor}{rgb}{1, 0, 0}
\newenvironment{knitrout}{}{} % an empty environment to be redefined in TeX

\usepackage{alltt}
\usepackage[margin=1.0in]{geometry} % To set margins
\usepackage{amsmath}  % This allows me to use the align functionality.
                      % If you find yourself trying to replicate
                      % something you found online, ensure you're
                      % loading the necessary packages!
\usepackage{amsfonts} % Math font
\usepackage{fancyvrb}
\usepackage{hyperref} % For including hyperlinks
\usepackage[shortlabels]{enumitem}% For enumerated lists with labels specified
                                  % We had to run tlmgr_install("enumitem") in R
\usepackage{float}    % For telling R where to put a table/figure
\usepackage{natbib}        %For the bibliography
\bibliographystyle{apalike}%For the bibliography
\IfFileExists{upquote.sty}{\usepackage{upquote}}{}
\begin{document}

\begin{enumerate}
%%%%%%%%%%%%%%%%%%%%%%%%%%%%%%%%%%%%%%%%%%%%%%%%%%%%%%%%%%%%%%%%%%%%%%%%%%%%%%%%
%%%%%%%%%%%%%%%%%%%%%%%%%%%%%%%%%%%%%%%%%%%%%%%%%%%%%%%%%%%%%%%%%%%%%%%%%%%%%%%%
% QUESTION 1
%%%%%%%%%%%%%%%%%%%%%%%%%%%%%%%%%%%%%%%%%%%%%%%%%%%%%%%%%%%%%%%%%%%%%%%%%%%%%%%%
%%%%%%%%%%%%%%%%%%%%%%%%%%%%%%%%%%%%%%%%%%%%%%%%%%%%%%%%%%%%%%%%%%%%%%%%%%%%%%%%
\item This week's Problem of the Week in Math is described as follows:
\begin{quotation}
  \textit{There are thirty positive integers less than 100 that share a certain 
  property. Your friend, Blake, wrote them down in the table to the left. But 
  Blake made a mistake! One of the numbers listed is wrong and should be replaced 
  with another. Which number is incorrect, what should it be replaced with, and 
  why?}
\end{quotation}
The numbers are listed below.
\begin{center}
  \begin{tabular}{ccccc}
    6 & 10 & 14 & 15 & 21\\
    22 & 26 & 33 & 34 & 35\\
    38 & 39 & 46 & 51 & 55\\
    57 & 58 & 62 & 65 & 69\\
    75 & 77 & 82 & 85 & 86\\
    87 & 91 & 93 & 94 & 95
  \end{tabular}
\end{center}
Use the fact that the ``certain'' property is that these numbers are all supposed
to be the product of \emph{unique} prime numbers to find and fix the mistake that
Blake made.\\
\textbf{Reminder:} Code your solution in an \texttt{R} script and copy it over
to this \texttt{.Rnw} file.\\
\textbf{Hint:} You may find the \verb|%in%| operator and the \verb|setdiff()| function to be helpful.\\

\textbf{Solution:} 
% Write your answer and explanations here.

\begin{knitrout}\scriptsize
\definecolor{shadecolor}{rgb}{0.969, 0.969, 0.969}\color{fgcolor}\begin{kframe}
\begin{alltt}
\hlcom{#################}
\hlcom{## Making set of prime numbers}
\hlcom{##################}
\hldef{prime} \hlkwb{=} \hlkwd{c}\hldef{(}\hlnum{2}\hldef{,} \hlnum{3}\hldef{,} \hlnum{5}\hldef{,} \hlnum{7}\hldef{,} \hlnum{11}\hldef{,} \hlnum{13}\hldef{,} \hlnum{17}\hldef{,} \hlnum{19}\hldef{,} \hlnum{23}\hldef{,} \hlnum{29}\hldef{,} \hlnum{31}\hldef{,} \hlnum{37}\hldef{,} \hlnum{41}\hldef{,} \hlnum{43}\hldef{,} \hlnum{47}\hldef{,} \hlnum{53}\hldef{,}
          \hlnum{59}\hldef{,} \hlnum{61}\hldef{,} \hlnum{67}\hldef{,} \hlnum{71}\hldef{,} \hlnum{73}\hldef{,} \hlnum{79}\hldef{,} \hlnum{83}\hldef{,} \hlnum{89}\hldef{,} \hlnum{97}\hldef{)} \hlcom{#Set of all prime numbers under 100}

\hldef{wrong.vec} \hlkwb{=} \hlkwd{c}\hldef{(}\hlnum{6}\hldef{,} \hlnum{10}\hldef{,} \hlnum{14}\hldef{,} \hlnum{15}\hldef{,} \hlnum{21}\hldef{,} \hlnum{22}\hldef{,} \hlnum{26}\hldef{,} \hlnum{33}\hldef{,} \hlnum{34}\hldef{,} \hlnum{35}\hldef{,} \hlnum{38}\hldef{,} \hlnum{39}\hldef{,} \hlnum{46}\hldef{,} \hlnum{51}\hldef{,} \hlnum{55}\hldef{,} \hlnum{57}\hldef{,} \hlnum{58}\hldef{,}
              \hlnum{62}\hldef{,} \hlnum{65}\hldef{,} \hlnum{69}\hldef{,} \hlnum{75}\hldef{,} \hlnum{77}\hldef{,} \hlnum{82}\hldef{,} \hlnum{85}\hldef{,} \hlnum{86}\hldef{,} \hlnum{87}\hldef{,} \hlnum{91}\hldef{,} \hlnum{93}\hldef{,} \hlnum{94}\hldef{,} \hlnum{95}\hldef{)}

\hldef{prime.mult} \hlkwb{=} \hlkwd{c}\hldef{()} \hlcom{#setting up what will be final vector}
\hldef{prime.index} \hlkwb{=} \hlkwd{c}\hldef{(}\hlnum{1}\hlopt{:}\hlkwd{length}\hldef{(prime))}

\hlcom{######################################}
\hlcom{##  Making a vector of unique prime multiples}
\hlcom{#######################################}

\hlkwa{for} \hldef{(i} \hlkwa{in} \hldef{prime.index)\{} \hlcom{#Loop over each prime to add itself and unique multiples to vector}
  \hldef{curr} \hlkwb{=} \hldef{prime[i]}
  \hlcom{#Initializing multiples}
  \hldef{count} \hlkwb{=} \hlnum{1}
  \hldef{mult} \hlkwb{=} \hldef{curr}\hlopt{*}\hldef{prime[i}\hlopt{+}\hldef{count]} \hlcom{#Initializing first multiple}
  \hlkwa{while}\hldef{(i}\hlopt{<}\hlkwd{length}\hldef{(prime)} \hlopt{&} \hldef{mult}\hlopt{<}\hlnum{100}\hldef{)\{} \hlcom{#Adding unique multiples}
    \hlkwa{if} \hldef{(}\hlopt{!}\hldef{mult} \hlopt \hldef{prime.mult)} \hlcom{#Checks if it is unique}
      \hldef{prime.mult} \hlkwb{=} \hlkwd{c}\hldef{(prime.mult, mult)} \hlcom{#Add to vector if it is unique}
    \hlcom{#Prepping for next iteration}
    \hldef{count} \hlkwb{=} \hldef{count}\hlopt{+}\hlnum{1}
    \hldef{mult} \hlkwb{=} \hldef{curr}\hlopt{*}\hldef{prime[i}\hlopt{+}\hldef{count]}
  \hldef{\}}
\hldef{\}}
\hlcom{###Making the vector sorted properly}
\hldef{prime.mult} \hlkwb{=} \hlkwd{sort}\hldef{(prime.mult)}

\hlcom{### Comparing given vs calculated vector}
\hlkwd{setdiff}\hldef{(wrong.vec, prime.mult)} \hlcom{#75 should NOT be in given vector}
\end{alltt}
\begin{verbatim}
## [1] 75
\end{verbatim}
\begin{alltt}
\hlkwd{setdiff}\hldef{(prime.mult, wrong.vec)} \hlcom{#74 SHOULD be in given vector}
\end{alltt}
\begin{verbatim}
## [1] 74
\end{verbatim}
\begin{alltt}
\hlcom{#From here, replacing either 74 with 75 in given vector}
\hlcom{#or replacing the given vector with prime.mult would successfully correct the mistake}
\end{alltt}
\end{kframe}
\end{knitrout}
\end{enumerate}

\bibliography{bibliography}
\end{document}
